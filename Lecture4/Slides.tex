\documentclass[10pt]{beamer}
\usepackage[cache=false]{minted}
\usepackage[utf8x]{inputenc}
\usepackage{hyperref}
\usepackage{fontawesome}
\usepackage{graphicx}
\usepackage[all, 2cell]{xy}
\usepackage[all]{xy}
\usepackage[english,ngerman]{babel}
\usepackage{bussproofs}

% ------------------------------------------------------------------------------
% Use the beautiful metropolis beamer template
% ------------------------------------------------------------------------------
\usepackage[T1]{fontenc}
\usepackage{fontawesome}
\usepackage{FiraSans}
\newtheorem{defin}{Definition}
\newtheorem{theor}{Theorem}
\newtheorem{prop}{Proposition}
\mode<presentation>
{
  \usetheme[progressbar=foot,numbering=fraction,background=light]{metropolis}
  \usecolortheme{default} % or try albatross, beaver, crane, ...
  \usefonttheme{default}  % or try serif, structurebold, ...
  \setbeamertemplate{navigation symbols}{}
  \setbeamertemplate{caption}[numbered]
  %\setbeamertemplate{frame footer}{My custom footer}
}

% ------------------------------------------------------------------------------
% beamer doesn't have texttt defined, but I usually want it anyway
% ------------------------------------------------------------------------------
\let\textttorig\texttt
\renewcommand<>{\texttt}[1]{%
  \only#2{\textttorig{#1}}%
}

% ------------------------------------------------------------------------------
% minted
% ------------------------------------------------------------------------------


% ------------------------------------------------------------------------------
% tcolorbox / tcblisting
% ------------------------------------------------------------------------------
\usepackage{xcolor}
\definecolor{codecolor}{HTML}{FFC300}

\usepackage{tcolorbox}
\tcbuselibrary{most,listingsutf8,minted}

\tcbset{tcbox width=auto,left=1mm,top=1mm,bottom=1mm,
right=1mm,boxsep=1mm,middle=1pt}

\newtcblisting{myr}[1]{colback=codecolor!5,colframe=codecolor!80!black,listing only,
minted options={numbers=left, style=tcblatex,fontsize=\tiny,breaklines,autogobble,linenos,numbersep=3mm},
left=5mm,enhanced,
title=#1, fonttitle=\bfseries,
listing engine=minted,minted language=r}


% ------------------------------------------------------------------------------
% Listings
% ------------------------------------------------------------------------------
\definecolor{mygreen}{HTML}{37980D}
\definecolor{myblue}{HTML}{0D089F}
\definecolor{myred}{HTML}{98290D}

\usepackage{listings}

% the following is optional to configure custom highlighting
\lstdefinelanguage{XML}
{
  morestring=[b]",
  morecomment=[s]{<!--}{-->},
  morestring=[s]{>}{<},
  morekeywords={ref,xmlns,version,type,canonicalRef,metr,real,target}% list your attributes here
}

\lstdefinestyle{myxml}{
language=XML,
showspaces=false,
showtabs=false,
basicstyle=\ttfamily,
columns=fullflexible,
breaklines=true,
showstringspaces=false,
breakatwhitespace=true,
escapeinside={(*@}{@*)},
basicstyle=\color{mygreen}\ttfamily,%\footnotesize,
stringstyle=\color{myred},
commentstyle=\color{myblue}\upshape,
keywordstyle=\color{myblue}\bfseries,
}

\title{Functional programming, Seminar No. 4}
\author{Danya Rogozin \\ Institute for Information Transmission Problems, RAS \\ Serokell O\"{U}}
\date{Higher School of Economics \\ The Faculty of Computer Science}
\begin{document}

\maketitle

\begin{frame}
  \frametitle{Today}

  We will study
  \begin{center}
  \includegraphics[scale=0.12]{mlem.jpeg}
  \end{center}
\end{frame}

\section{Algebraic data types and pattern matching}

\begin{frame}[fragile]
\frametitle{Pattern matching}
  Let us take a look at the following functions:
  \begin{minted}{haskell}
  swap :: (a, b) -> (b, a)
  swap (a, b) = (b, a)

  length :: [a] -> Int
  length [] = 0
  lenght (x : xs) = 1 + length xs
  \end{minted}

\onslide<2->{
  \begin{itemize}
    \item Terms like \verb"(a,b)", \verb"[]", and \verb"(x : xs)" are called \emph{patterns}
    \item One needs to check whether the constructors \verb"(,)" and \verb"( : )" are relevant.
    \item Consider \verb"swap (45, True)". Variables \verb"a" and \verb"b" are bound with the values \verb"45" and \verb"True".
    \item Consider \verb"lenght [1,2,3]". Variables \verb"x" and \verb"xs" are bound with the values \verb"1" and \verb"[2,3]"
  \end{itemize}
  }
\end{frame}

\begin{frame}[fragile]
\frametitle{Algebraic data types. Sums}

The simplest example of an algebraic data type is a data type defined with an enumeration of constructors that stores no values.
\begin{minted}{haskell}
data Colour = Red | Blue | Green | Purple | Yellow
  deriving (Show, Eq)

isRGB :: Colour -> Bool
isRGB Red  = True
isRGB Blue = True
isRGB Green = True
isRGB _     = False       -- Wild-card
\end{minted}
\end{frame}

\begin{frame}[fragile]
\frametitle{Algebraic data types. Products}

\begin{itemize}
\item An example of a product data type:
\begin{minted}{haskell}
data Point = Point Double Double
  deriving Show

> :type Point
Point :: Double -> Double -> Point
\end{minted}
\item An example of a function
\begin{minted}{haskell}
taxiCab :: Point -> Point -> Double
taxiCab (Point x1 y1) (Point x2 y2) =
  abs (x1 - x2) + abs (y1 - y2)
\end{minted}
\end{itemize}
\end{frame}

\begin{frame}[fragile]
  \frametitle{Polymorphic data types}

\begin{itemize}
  \item That point data type might be parametrised with a type parameter:
  \begin{minted}{haskell}
  data Point a = Point a a
    deriving Show
  \end{minted}

  \item The \verb"Point" data constructor has the following type. The \verb"Point" from the left (see the definition above) is a type function that has its type (kind).
  \begin{minted}{haskell}
  > :type Point
  Point :: a -> a -> Point a
  > :kind Point
  Point :: * -> *
  \end{minted}
\end{itemize}
\end{frame}

\begin{frame}[fragile]
  \frametitle{Polymorphic data types and type classes}

\begin{itemize}
  \item Suppose we have a function:
  \begin{minted}{haskell}
 midPoint
   :: Fractional a => Point a -> Point a -> Point a
 midPoint (Pt x1 y1) (Pt x2 y2) =
   Pt ((x1 + x2) / 2) ((y1 + y2) / 2)
  \end{minted}
  \item Playing with GHCi:
  \begin{minted}{haskell}
  > :t midPoint (Pt 3 5) (Pt 6 4)
  midPoint (Pt 3 5) (Pt 6 4) :: Fractional a => Point a
  > midPoint (Pt 3 5) (Pt 6 4)
  Pt 4.5 4.5
  > :t it
  it :: Fractional a => Point a
  \end{minted}
\end{itemize}
\end{frame}

\begin{frame}[fragile]
  \frametitle{Inductive data types}

\begin{itemize}
  \item The list is the first example of an inductive data type
  \begin{minted}{haskell}
  data List a = Nil | Cons a (List a)
    deriving Show
  \end{minted}
  \item The data constructors are \verb"Nil :: List a" and \verb"Cons :: a -> List a -> List a"
  \item Pattern matching and recursion
  \begin{minted}{haskell}
  concat :: List a -> List a -> List a
  concat Nil ys = ys
  concat (Cons x xs) ys = Cons x (xs `concat` ys)
  \end{minted}
\end{itemize}
\end{frame}

\begin{frame}[fragile]
  \frametitle{Standard lists}

  \begin{itemize}
    \item The list data type is already in the standard library, but its approximate definition is the following one:
    \begin{minted}{haskell}
    infixr 5 :
    data [a] = [] | a : [a]
      deriving Show
    \end{minted}
    \item Syntax sugar:
    \begin{minted}{haskell}
    [1, 2, 3, 4] == 1 : 2 : 3 : 4 : []
    \end{minted}
    \item The example of a definition with built-in lists:
    \begin{minted}{haskell}
    infixr 5  ++
    (++) :: [a] -> [a] -> [a]
    (++) []     ys = ys
    (++) (x:xs) ys = x : xs ++ ys
  \end{minted}
  \end{itemize}
\end{frame}

\begin{frame}[fragile]
  \frametitle{\verb"case-of" expressions}
  \begin{itemize}
    \item \verb"case ... of ..." expressions allows one to patternmatch everywhere
    \begin{minted}{haskell}
    filter :: (a -> Bool) -> [a] -> [a]
    filter p [] = []
    filter p (x : xs) =
      case p x of
        True  -> x : filter p xs
        False -> filter p xs
    \end{minted}
    \item The pattern matching from the previous slide is a syntax sugar for the corresponding \verb"case ... of ..." expression
  \end{itemize}
\end{frame}

\begin{frame}[fragile]
  \frametitle{Semantic aspects of pattern matching}
  \begin{itemize}
    \item Pattern matching is performed from up to down and from left to right after that.
    \item A pattern match is either
    \begin{itemize}
      \item successfully matches, or
      \item fails.
    \end{itemize}
    \item Block of patterns is either
    \begin{itemize}
      \item exhaustive (a successful match always exists), or
      \item inexhaustive (all matches can fail).
    \end{itemize}
    \item Here is an example:
    \begin{minted}{haskell}
    foo (1,4) = 7
    foo (0,_) = 8
    \end{minted}
    \item \verb"(0, undefined)" fails in the first case and it succeeds in the second one
    \item \verb"(undefined, 0)" diverges during a match
    \item What about \verb"(1,7-3)"?
  \end{itemize}
\end{frame}

\begin{frame}[fragile]
  \frametitle{As-patterns}
  \begin{itemize}
  \item Suppose we have the following function
  \begin{minted}{haskell}
  dupHead :: [a] -> [a]
  dupHead (x : xs) = x : x : xs
  \end{minted}
  \item One may rewrite this function as follows:
  \begin{minted}{haskell}
  dupHead :: [a] -> [a]
  dupHead s@(x : xs) = x : s
  \end{minted}
  \item Here, the name \verb"s" is assigned to the whole pattern \verb"x : xs"
\end{itemize}
\end{frame}

\begin{frame}[fragile]
  \frametitle{Irrefutable patterns}
  \begin{itemize}
    \item Irrefutable patterns are wild-cards, variables, and lazy patterns
    \item An example of a lazy pattern:
    \begin{minted}{haskell}
    > f *** g (a,b) = (f a, g b)
    > (const 2) *** (const 1) $ undefined
    *** Exception: Prelude.undefined
    > f *** g ~(a,b) = (f a, g b)
    > (const 2) *** (const 1) $ undefined
    (2,1)
    \end{minted}
  \end{itemize}
\end{frame}

\begin{frame}[fragile]
\frametitle{\verb"newtype" and \verb"type" declarations}

\begin{itemize}
  \item The keyword \verb"type" introduces type synonyms.
  \begin{minted}{haskell}
  type String = [Char]
  \end{minted}
  \item In Haskell, the string data type type is merely a type synonym for the list of characters
  \item The keyword \verb"newtype" defines a new type with the single constructor that packs a value of a given type
  \begin{minted}{haskell}
  newtype Age = Age Int
  \end{minted}

  \item The same type \verb"Age" defined with the accessor \verb"runAge"
  \begin{minted}{haskell}
  newtype Age = Age { runAge :: Int }
  -- where runAge :: Age -> Int
  \end{minted}
\end{itemize}
\end{frame}

\begin{frame}[fragile]
  \frametitle{Field labels}
  \begin{itemize}
    \item Sometimes product data types are too cumbersome:
    \begin{minted}{haskell}
    data Person = Person String String Int Float String
    \end{minted}
    \item As an alternative, one may define a data type with field labels
    \begin{minted}{haskell}
    data Person =
      Person { firstName :: String
             , lastName :: String
             , age :: Int
             , height :: Float
             , phoneNumber :: String
             }
     \end{minted}
     \item Such a data type is a record with accessors such as
     \verb"firstName :: Person -> String"
  \end{itemize}
\end{frame}

\begin{frame}[fragile]
  \frametitle{Field labels and type classes}
  \begin{itemize}
    \item Let us recall the \verb"Eq" type class once more
    \begin{minted}{haskell}
    class Eq a where
      (==)  :: a -> a -> Bool
      (/=)  :: a -> a -> Bool

    instance Eq Int where
      x == y = x `eqInt` y

    eqFunction :: Eq a => a -> a -> Int
    eqFunction x y =
      case x == y of
        True  -> 42
        False -> 0
    \end{minted}
    \item In fact, type classes are sugar for data types with field labels
    \item The constraint \verb"Eq a" is an additional argument
  \end{itemize}
\end{frame}

\begin{frame}[fragile]
  \frametitle{Field labels and type classes}

  \begin{itemize}
    \item The previous listing a bit unsugared (very roughly):
    \begin{minted}{haskell}
    data Eq a =
      Eq { eq :: a -> a -> Bool
         , neq :: a -> a -> Bool
         }

    intInstance :: Eq Int
    intInstance = Eq eqInt (\x y -> not $ x `eqInt` y)

    eqFunction :: Eq a -> a -> a -> Int
    eqFunction eqInst x y =
      case ((eq eqInst) x y) of
        True -> 42
        False -> 0
    \end{minted}
  \end{itemize}
\end{frame}

\begin{frame}[fragile]
  \frametitle{Some standard algebraic data types}
  \begin{itemize}
    \item The \verb"Maybe a" data type allows one to define an optional value:
    \begin{minted}{haskell}
    data Maybe a = Nothing | Just a

    maybe :: b -> (a -> b) -> Maybe a -> b
    maybe b _ Nothing = b
    maybe b f (Just x) = f x
    \end{minted}
    \item A simple example
    \begin{minted}{haskell}
    safeHead :: [a] -> Maybe a
    safeHead [] = Nothing
    safeHead (x : _) = Just x
    \end{minted}
  \end{itemize}
\end{frame}

\begin{frame}[fragile]
  \frametitle{Some standard algebraic data types}
  \begin{itemize}
    \item The \verb"Either" data type describes one or the other value
    \begin{minted}{haskell}
    data Either e a = Left e | Right a

    either :: (a -> c) -> (b -> c) -> Either a b -> c
    either f _ (Left x) = f x
    either _ g (Right x) = g x
    \end{minted}
    \item An example:
    \begin{minted}{haskell}
    safeTail :: [a] -> Either String [a]
    safeTail [] = Left "I have no tail, mate"
    safeTail (_ : xs) = Right xs
    \end{minted}
  \end{itemize}
\end{frame}

\section{Folds}

\begin{frame}[fragile]
  \frametitle{Folds and lists. Motivation}

Take a look at these functions
  \begin{minted}{haskell}
  sum :: Num a => [a] -> a
  sum [] = 0
  sum (x : xs) = x + sum xs

  product :: Num a => [a] -> a
  product [] = 1
  product (x : xs) = x * product xs

  concat :: [[a]] -> [a]
  concat [] = []
  concat (x : xs) = x ++ concat xs
  \end{minted}
\end{frame}

\begin{frame}[fragile]
\frametitle{The definition of a right fold}
\begin{itemize}
  \item The definition of a right is the following one
  \begin{minted}{haskell}
  foldr :: (a -> b -> b) -> b -> [a] -> b
  foldr _ ini [] = []
  foldr f ini (x : xs) = f x (foldr f ini xs)
  \end{minted}
  \item An informal explanation:
  \begin{minted}{haskell}
  foldr f z [x1, x2, ..., xn] ==
    x1 `f` (x2 `f` ... (xn `f` z)...)
  \end{minted}
\end{itemize}
\end{frame}

\begin{frame}
  \frametitle{The definition of a right fold}
One may visualise that for some list \verb"[a,b,c]". The list from the left and its right fold from the right
\begin{small}
\xymatrix{
 & \verb":" \ar[dl] \ar[dr] &&&&& \verb"f" \ar[dl] \ar[dr] \\
 \verb"a" && \verb":" \ar[dl] \ar[dr] &&& \verb"a" && \verb"f" \ar[dl] \ar[dr] \\
 & \verb"b" && \verb":" \ar[dl] \ar[dr] &&& \verb"b" && \verb"f"  \ar[dl] \ar[dr] \\
 && \verb"c" && \verb"[]" &&& \verb"c" && \verb"ini"
}
\end{small}
\end{frame}

\begin{frame}[fragile]
  \frametitle{Functions \verb"sum", \verb"product", and \verb"concat" with \verb"foldr"}
    \begin{minted}{haskell}
    sum :: Num a => [a] -> a
    sum = foldr (+) 0

    product :: Num a => [a] -> a
    product = foldr (*) 1

    concat :: [[a]] -> [a]
    concat = foldr (++) []
    \end{minted}
\end{frame}

\begin{frame}[fragile]
  \frametitle{The universal property of a right fold}
\begin{block}{The universal property}
  Let \verb"g" be a function defined by the following equations:
  \begin{minted}{haskell}
  g [] = v
  g (x : xs) = f x (g xs)
  \end{minted}

  then one has $\forall \: \verb"xs :: [a]" \:\: (\verb"g xs" \equiv \verb"foldr f v xs")$
\end{block}
\begin{itemize}
  \item The universal property is proved inductively
  \item This property implies \verb"foldr f v" and \verb"g" are equivalent in this case
\end{itemize}
\end{frame}

\begin{frame}[fragile]
  \frametitle{The definition of a left fold}
  \begin{itemize}
    \item In addition to the right fold, one also has the left one
    \begin{minted}{haskell}
    foldl :: (b -> a -> b) -> b -> [a] -> b
    foldl _ ini [] = ini
    foldl f ini (x : xs) = foldl f (f ini x) xs
    \end{minted}
    \item Informally:
    \begin{minted}{haskell}
    foldl f ini [x1, x2, ..., xn]
      == (...((ini `f` x1) `f` x2) `f`...) `f` xn
    \end{minted}
    \onslide<2->{
    \item The implementation of the left fold function might be optimised.
    \item \verb"foldl" is the most optimal function, but we are not capable of processing infinite lists using the left fold function.
    }
  \end{itemize}
\end{frame}

\begin{frame}[fragile]
  \frametitle{Are \verb"foldr" and \verb"foldl" equivalent?}

\begin{itemize}
  \item Note that \verb"foldr" and \verb"foldl" are not equivalent to each other
  \begin{minted}{haskell}
  > foldl (/) 64 [4,2,4]
  2.0
  > foldr (/) 64 [4,2,4]
  0.125
  > foldl (\x y -> 2*x + y) 4 [1,2,3]
  43
  > foldr (\x y -> 2*x + y) 4 [1,2,3]
  16
  \end{minted}
  \item \verb"foldr" and \verb"foldl" are equivalent if the folding operation is commutative
\end{itemize}
\end{frame}

\begin{frame}[fragile]
  \frametitle{The right scan}
  \begin{itemize}
    \item The right scan is the foldr that yields a list that contains all intermediate values
    \begin{minted}{haskell}
    scanr :: (a -> b -> b) -> b -> [a] -> [b]
    scanr _ ini [] = [ini]
    scanr f ini (x:xs) = f x q : qs
      where qs@(q:_) = scanr f ini xs
    \end{minted}
    \item \verb"foldr" and \verb"scanr" are connected with each other as follows
    \begin{center}
      $\verb"head (scanr f z xs)" \equiv \verb"foldr f z xs"$
    \end{center}
    \item The examples are
    \begin{minted}{haskell}
    > scanr (:) [] [1,2,3]
    [[1,2,3],[2,3],[3],[]]
    > scanr (+) 0 [1..10]
    [55,54,52,49,45,40,34,27,19,10,0]
    > scanr (*) 1 [1..5]
    [120,120,60,20,5,1]
    \end{minted}
  \end{itemize}
\end{frame}

\begin{frame}[fragile]
  \frametitle{The left scan}
  \begin{itemize}
  \item One also has a scan function for the \verb"foldl" function:
  \begin{minted}{haskell}
  scanl :: (b -> a -> b) -> b -> [a] -> [b]
  scanl f q ls = q : (case ls of
                        []   -> []
                        x:xs -> scanl f (f q x) xs)
  \end{minted}
  \item \verb"foldl" and \verb"scanl" are connected with each other as follows:
  \begin{center}
   $\verb"last (scanl f z xs)" \equiv \verb"foldl f z xs"$
  \end{center}
  \item The examples:
  \begin{minted}{haskell}
  > scanl (++) "!" ["a","b","c"]
  ["!","!a","!ab","!abc"]
  > scanl (*) 1 [1..] !! 5
  120
  \end{minted}
  \item In contrast to \verb"foldl", \verb"scanl" works with infinite lists.
  \end{itemize}
\end{frame}

\section{Strictness in Haskell}

\begin{frame}[fragile]
  \frametitle{Bottom}

  \begin{itemize}
    \item Any well-formed expression in Haskell has a type
    \item Prima facie, the \verb"Bool" data type has two values: \verb"False" and \verb"True" according to its definition:
    \begin{minted}{haskell}
    data Bool = False | True
    \end{minted}
    \item One may define an expession \verb"dno :: Bool" which is defined recursively as \verb"dno = not dno"
    \item \verb"dno" is neither \verb"False" nor \verb"True", but it's a Boolean value!
    \item This value is a bottom ($\bot$). In Haskell, $\bot$ is a value that has a type \verb"forall a. a".
    Such errors as \verb"undefined" have this type.
  \end{itemize}
\end{frame}

\begin{frame}
  \frametitle{Strict functions}
  \begin{itemize}
    \item Haskell is lazy. That's why \verb"const 42 undefined == 42"
    \item Lazy functions are non-strict ones
    \onslide<2->{
    \item In constrast to lazy functions, strict functions satisfy this equation
    \begin{center}
      $\verb"f" \: x_1 \: x_2 \: \dots \: \bot \: \dots \: x_n = \bot$
    \end{center}
    \item For this reason \verb"constStrict 42 undefined = undefined"
    }
  \end{itemize}
\end{frame}

\begin{frame}[fragile]
  \frametitle{Strictness in Haskell. The \verb"seq" function}
\begin{itemize}
  \item We've already had a look at the \verb"seq" function.
  \item \verb"seq" is a combinator that enforces computation. It evaluates the first argument to its WHNF.
  \item This combinator has a type $a \to b \to b$.
  \item It's quite close to something like $\lambda x y. y$, but \verb"seq" satisfies the following equations:
  \begin{center}
    $\verb"seq" \: \bot \: x = \bot$

    $\verb"seq" \: v \: x = x, v \neq \bot $
  \end{center}
  \item This function ``breaks'' our laziness! But this enforcing with \verb"seq" is not so far-reaching.
  Data constructors and lambdas put a barrier for the $\bot$ expansion:
  \begin{minted}{haskell}
  > seq (4,undefined) 5
  5
  > seq (\x -> undefined) 5
  5
  > seq (id . undefined) 5
  5
  \end{minted}
  \item The library \verb"deepseq" contains the same titled combinator that \emph{fully} evaluates the first argument.
\end{itemize}
\end{frame}

\begin{frame}[fragile]
  \frametitle{Strictness in Haskell. The strict application}
  \begin{itemize}
    \item One may implement the strict appication using \verb"seq"
    \begin{minted}{haskell}
    infixr 0 $!
    ($!) :: (a -> b) -> a -> b
    f $! x = x `seq` f x
    \end{minted}
    \item That is, this application behaves as usual unless the second argument is the bottom.
  \end{itemize}
\end{frame}

\begin{frame}[fragile]
  \frametitle{Strictness in Haskell. The strict application}
  \begin{itemize}
    \item Let us recall the tail-recursive factorial. The second version is strict:
    \begin{minted}{haskell}
      tailFactorial :: Integer -> Integer
      tailFactorial n = helper 1 n
        where
        helper acc x =
          if x > 1
          then helper (acc * x) (x - 1)
          else acc

      tailFactorialStrict :: Integer -> Integer
      tailFactorialStrict n = helper 1 n
        where
          helper acc x =
            if x > 1
            then (helper $! (acc * x)) (x - 1)
            else acc
      \end{minted}
  \end{itemize}
\end{frame}

\begin{frame}[fragile]
  \frametitle{The strict \verb"foldl"}

  \begin{itemize}
    \item The strict version of \verb"foldl"
    \begin{minted}{haskell}
    foldl' :: (a -> b -> a) -> a -> [b] -> a
    foldl' f ini [] = ini
    foldl' f ini (x:xs) = foldl' f arg xs
      where arg = (f ini) $! x
    \end{minted}
  \end{itemize}
\end{frame}

\begin{frame}[fragile]
  \frametitle{Strictness in Haskell. Bang patterns}
  \begin{itemize}
    \item A data type might contain strict values with the strictness flag \verb"!", e.g.
    \begin{minted}{haskell}
    data Complex a = !a :+ !a
      deriving Show
    infix 6 :+

    im :: Complex a -> a
    im (x :+ y) = y
    \end{minted}
    > im (undefined :+ 5)
    *** Exception: Prelude.undefined
    \begin{lstlisting}[language=Haskell]
    \end{lstlisting}

  \item The \verb"BangPatterns" extension allows one to make pattern a strict one
  \begin{minted}{haskell}
  > :set -XBangPatterns
  > foo !x = True
  > foo undefined
  *** Exception: Prelude.undefined
  \end{minted}
  \end{itemize}
\end{frame}

\begin{frame}
  \frametitle{Summary}

  Today we
  \begin{itemize}
    \item discussed the data type landscape and together with pattern matching
    \item studied folds
    \item realised how one can enforce lazy evaluation
  \end{itemize}

  \onslide<2->{
  On the next seminar, we will
  \begin{itemize}
    \item study such type classes as \verb"Functor", \verb"Foldable", and \verb"Monoid"
  \end{itemize}
  }
\end{frame}

\end{document}
