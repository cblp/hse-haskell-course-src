\documentclass[10pt]{beamer}
\usepackage[cache=false]{minted}
\usepackage[utf8x]{inputenc}
\usepackage{hyperref}
\usepackage{fontawesome}
\usepackage{graphicx}
\usepackage[english,ngerman]{babel}
\usepackage{bussproofs}

% ------------------------------------------------------------------------------
% Use the beautiful metropolis beamer template
% ------------------------------------------------------------------------------
\usepackage[T1]{fontenc}
\usepackage{fontawesome}
\usepackage{FiraSans}
\newtheorem{defin}{Definition}
\newtheorem{theor}{Theorem}
\newtheorem{prop}{Proposition}
\mode<presentation>
{
  \usetheme[progressbar=foot,numbering=fraction,background=light]{metropolis}
  \usecolortheme{default} % or try albatross, beaver, crane, ...
  \usefonttheme{default}  % or try serif, structurebold, ...
  \setbeamertemplate{navigation symbols}{}
  \setbeamertemplate{caption}[numbered]
  %\setbeamertemplate{frame footer}{My custom footer}
}

% ------------------------------------------------------------------------------
% beamer doesn't have texttt defined, but I usually want it anyway
% ------------------------------------------------------------------------------
\let\textttorig\texttt
\renewcommand<>{\texttt}[1]{%
  \only#2{\textttorig{#1}}%
}

% ------------------------------------------------------------------------------
% minted
% ------------------------------------------------------------------------------


% ------------------------------------------------------------------------------
% tcolorbox / tcblisting
% ------------------------------------------------------------------------------
\usepackage{xcolor}
\definecolor{codecolor}{HTML}{FFC300}

\usepackage{tcolorbox}
\tcbuselibrary{most,listingsutf8,minted}

\tcbset{tcbox width=auto,left=1mm,top=1mm,bottom=1mm,
right=1mm,boxsep=1mm,middle=1pt}

\newtcblisting{myr}[1]{colback=codecolor!5,colframe=codecolor!80!black,listing only,
minted options={numbers=left, style=tcblatex,fontsize=\tiny,breaklines,autogobble,linenos,numbersep=3mm},
left=5mm,enhanced,
title=#1, fonttitle=\bfseries,
listing engine=minted,minted language=r}


% ------------------------------------------------------------------------------
% Listings
% ------------------------------------------------------------------------------
\definecolor{mygreen}{HTML}{37980D}
\definecolor{myblue}{HTML}{0D089F}
\definecolor{myred}{HTML}{98290D}

\usepackage{listings}

% the following is optional to configure custom highlighting
\lstdefinelanguage{XML}
{
  morestring=[b]",
  morecomment=[s]{<!--}{-->},
  morestring=[s]{>}{<},
  morekeywords={ref,xmlns,version,type,canonicalRef,metr,real,target}% list your attributes here
}

\lstdefinestyle{myxml}{
language=XML,
showspaces=false,
showtabs=false,
basicstyle=\ttfamily,
columns=fullflexible,
breaklines=true,
showstringspaces=false,
breakatwhitespace=true,
escapeinside={(*@}{@*)},
basicstyle=\color{mygreen}\ttfamily,%\footnotesize,
stringstyle=\color{myred},
commentstyle=\color{myblue}\upshape,
keywordstyle=\color{myblue}\bfseries,
}


% ------------------------------------------------------------------------------
% The Document
% ------------------------------------------------------------------------------
\title{Functional programming, Seminar No. 1}
\author{Danya Rogozin \\ Institute for Information Transmission Problems, RAS \\ Serokell O\"{U}}

\vspace{\baselineskip}

\date{Higher School of Economics \\ The Department of Computer Science}

\begin{document}
\maketitle

\section{General words on Haskell and History}

\begin{frame}
  \frametitle{Intro}

  \begin{itemize}
    \item The language is named after Haskell Curry, an American logician
    \item The first implementation: 1990
    \item The language standard: Haskell2010
    \item Default compiler: Glasgow Haskell compiler
    \item Haskell is a strongly-typed, polymorphic, and purely functional programming language
    \onslide<2->{
    \item This course is quite introductory.
    }
    \onslide<3->{
    \item Vox populi:
    \begin{center}
    \includegraphics[scale=0.5]{Pics/VoxPopuli.png}
    \end{center}
    }
  \end{itemize}
\end{frame}

\begin{frame}
  \frametitle{Lambda calculus and type theory. Incomplete and Utter History of Functional Programming}

\onslide<1->{
  \begin{itemize}
    \item At the end of the 1920-s, Alonzo Church proposed an alternative approach to the foundations of mathematics where the notion of a function is the primitive one. The lambda calculus is a formal system that describes arbitrary abstract functions.
    }
\onslide<2->{
    \item Moreover, Church used the lambda calculus to show that Peano arithmetic is undecidable.
    }
  \onslide<3->{
    \item Kleene and Rosser showed that the initial version of the lambda calculus is inconsistent. Initially, the idea of typing was the instrument that would allow us to avoid paradoxes.
  }
  \onslide<4->{
    \item The first system of typed the lambda calculus is a hybrid from the lambda calculus and type theory developed by Bertrand Russell and Alfred North Whitehead (1910-s).
  }
  \end{itemize}
\end{frame}

\begin{frame}
  \frametitle{Lambda calculus and type theory. Historical notes}

\onslide<1->{
  \begin{itemize}
    \item After Church’s works, type theory as the branch of $\lambda$ calculus and combinatory logic was developed by Haskell Curry and William Howard within the context of proof theory (1950-1960-s)
    }
  \onslide<2->{
    \item Polymorphic lambda calculus (John Reynolds and Jean-Yves Girard (1970-s))
    }
  \onslide<3->{
    \item Polymorphic type inference (Roger Hindley, Robin Milner and Luis Damas (1970-1980-s))
    }
  \onslide<4->{
    \item ML: the very first language with polymorphic inferred type system (Robin Milner, 1973)
    }
  \onslide<5->{
    \item The language Haskell appeared at the beginning of 1990-s. Haskell desinged by Simon Peyton Jones, Philip Wadler, and others
    }
  \end{itemize}
\end{frame}

\begin{frame}
  \frametitle{Functional programming and its foundations}
  \onslide<1->{
  The lambda calculus establishes the foundations for functional programming in the same manner as the von Neumann principles are the root of imperative programming.
  } \onslide<2->{
  \begin{itemize}
    \item We have no assignment as in imperative languages.  Variables are nullary constant functions rather than so-called boxes.
    \item Hence, we have no states
    \item We use recursion instead of loops
    \item Pattern-matching
  \end{itemize}
  }
\end{frame}

\begin{frame}
  \frametitle{What are types needed for?}

\metroset{block=fill}
\begin{exampleblock}{According to Benjamin Pierce}
A type system is a tractable syntactic method for proving the absence of certain program behaviours by classifying phrases according to the kinds of values they compute.
\end{exampleblock}

\onslide<2->{
It's about
  \begin{itemize}
    \item A partial specification
    \item Type preserving
    \item Type checking allows one to catch simple errors
    \item Type inference
    \item Etc
  \end{itemize}
  }
\end{frame}

\begin{frame}
  \frametitle{A landscape of typing from a bird's eye view}
We may classify possible ways of typing as follows

\begin{itemize}
  \item Static and dynamic typing
  \begin{itemize}
    \item C, C++, Java, Haskell, etc
    \item JavaScript, Ruby, PHP, etc
  \end{itemize}
  \item Implicit and explicit typing
  \begin{itemize}
    \item JavaScript, Ruby, PHP, etc
    \item C++, Java, etc
  \end{itemize}
  \item Inferred typing
  \begin{itemize}
    \item Haskell, Standard ML, Ocaml, Idris, etc
  \end{itemize}
\end{itemize}

\end{frame}

\section{Ecosystem}

\begin{frame}
  \frametitle{The Haskell Platform installation}

  There are several ways to install the Haskell platform on Mac:

\onslide<1->{
  \begin{itemize}
    \item Download the \verb".pkg" file and install the corresponding package}
    \onslide<2->{\item Run the script
\begin{center}
    \verb"curl -sSL https://get.haskellstack.org/ | sh"
\end{center}
    }
    \onslide<3->{\item Install ghc, stack, and cabal using Homebrew
  \end{itemize}
  }

\onslide<4->{
  Choose any way you prefer. All these ways are equivalent to each other.
}
  \vspace{\baselineskip}


\onslide<5->{
 I'm a Mac user, but I believe that you'll manage to install the Haskell Platform on NixOs/Windows/Linux/etc quite quickly.
 }
\end{frame}

\begin{frame}
  \frametitle{GHC}

\onslide<1->{
  \begin{itemize}
    \item GHC is a default Haskell compiler.
    \item GHC is an open-source project. Don't hesitate to contribute!
    \item GHC is mostly implemented on Haskell.
    }
    \item GHC is developed under the GHC Steering committee control.
    \onslide<2->{
    \item Very roughly, compiling pipeline is arranged as follows:
\begin{center}
    parsing $\Rightarrow$ compile-time (type-checking mostly)
    $\Rightarrow$ runtime (program execution)
\end{center}
    }
  \end{itemize}
\end{frame}

\begin{frame}
  \frametitle{GHCi}

  \begin{itemize}
    \item GHCi is a Haskell interpreter based on GHC.
    \item One may run GHCi with the command \verb"ghci".
    \item You may play with GHCi as a calculator, the ordinary arithmetic operators are usual
    \item You may also have a look at the GHCi chapter in the GHC User's Guide to get familiar with GHCi closely.
  \end{itemize}

  \begin{center}
  \includegraphics[scale=0.46]{Pics/GHCi.png}
  \end{center}
\end{frame}

\begin{frame}
  \frametitle{Cabal}

\onslide<1->{
  \begin{itemize}
    \item Cabal is a system of library and dependency management
    \item A \verb".cabal" file describes the version of a package and its dependencies
    \item Cabal is also a packaging tool
    \item Cabal used to cause dependency hell
  \end{itemize}}
\end{frame}

\begin{frame}
  \frametitle{Stack}

\onslide<1->{
  \begin{itemize}
    \item Stack is a \emph{mainstream} cross-platform build tool for Haskell projects
    \item Stack is about
    }
  \onslide<2->{
    \begin{itemize}
      \item installation of required packages and the latest GHC (and their more concrete versions),
      \item building, execution, and testing
      \item creating an isolated location.
      \item Builds are reproducible
    \end{itemize}
  \end{itemize}
  }
\end{frame}

\begin{frame}
  \frametitle{Snapshots}

  \begin{itemize}
    \onslide<1->{
    \item A \emph{snapshot} is a curated package set used by Stack
    }
    \onslide<2->{
    \item Stackage is a stable repository that stores snapshots}
    \onslide<3->{\item A \emph{resolver} is a reference to a required snapshot
    }
    \onslide<4->{
    \item A screenshot from Stackage:

    \begin{center}
    \includegraphics[scale=0.4]{Pics/Snapshots.png}
    \end{center}
    }
  \end{itemize}
\end{frame}

\begin{frame}
  \frametitle{Ecosystem encapsulation}

  The Haskell ecosystem encapsulation might be described as the sequence of the following inclusions:

  \begin{center}
  \includegraphics[scale=0.15]{Pics/Eco.jpeg}
  \end{center}
\end{frame}

\begin{frame}
  \frametitle{Creating a Haskell project with Stack}
  \onslide<1->{
  \begin{itemize}
    \item Figure out how to call your project and run the script \verb"stack new <projectname>"
    \item You will see the following story after the command \verb"tree ." in the project directory:
    }
  \onslide<2->{

  \begin{center}
  \includegraphics[scale=0.35]{Pics/Tree.png}
  \end{center}
  \end{itemize}
  }
\end{frame}

\begin{frame}
  \frametitle{\verb"stack.yaml"}

\onslide<1->{
  Let us discuss on how dependency files look like. First of all, we observe the \verb"stack.yaml" file:
}

\onslide<2->{
\begin{center}
\includegraphics[scale=0.35]{Pics/StackYaml.png}
\end{center}
}
\end{frame}

\begin{frame}
  \frametitle{Cabal file}

\onslide<1->{
  As we told above, the \verb".cabal" file describes the relevant version of a project and its dependencies:
}
\onslide<2->{
\begin{center}
\includegraphics[scale=0.273]{Pics/CabalFile.png}
\end{center}
  }
\end{frame}

\begin{frame}
  \frametitle{\verb"package.yaml"}

\onslide<1->{
  The \verb"package.yaml" is used to generate the \verb".cabal" file automatically:
}
\onslide<2->{
\begin{center}
\includegraphics[scale=0.25]{Pics/PackageYaml.png}
\end{center}
}
\end{frame}

\begin{frame}
  \frametitle{Building and running a project}

The basic commands:
  \begin{itemize}
    \item \verb"stack build"
    \item \verb"stack run"
    \item \verb"stack exec"
    \item \verb"stack ghci"
    \item \verb"stack clean"
    \item \verb"stack test"
  \end{itemize}
\onslide<2->{
  The roles of all these commands follow from their quite self-explanatory names.
  }
\end{frame}

\begin{frame}
  \frametitle{Hackage}
  According to its description, 'Hackage is the Haskell community's central package archive of open source software`.

\onslide<2->{
  \begin{itemize}
    \item Webpage: \verb"https://hackage.haskell.org"
    }
\onslide<3->{
    \item Browsing packages, simplified package search, current uploads.
  \end{itemize}
  }
\onslide<4->{
\begin{center}
\includegraphics[scale=0.3]{Pics/HackageExample.png}
\end{center}
  }
\end{frame}

\begin{frame}
  \frametitle{Hoogle}

\onslide<1->{
  Hoogle is a sort of Haskell search engine. Webpage: \verb"https://hoogle.haskell.org".
}

\onslide<2->{
\begin{center}
\includegraphics[scale=0.24]{Pics/Hoogle.png}
\end{center}
}
\end{frame}

\begin{frame}
  \frametitle{Hackage Search}

  Hackage Search is a searching tool for Hackage based on regular expressions. This tool is by Vlad Zavialov, my GHC teammate from Serokell.

  \verb"https://hackage-search.serokell.io".

\onslide<2->{
\begin{center}
\includegraphics[scale=0.25]{Pics/HackageSearch1.png}
\end{center}
}
\end{frame}

\begin{frame}
  \frametitle{Hackage Search}

\onslide<1->{
  Hackage Search is a searching tool for Hackage based on regular expressions. This tool is made Vlad Zavialov, my GHC teammate from Serokell.

  \verb"https://hackage-search.serokell.io".
}
\begin{center}
\includegraphics[scale=0.24]{Pics/HackageSearch2.png}
\end{center}
\end{frame}

\begin{frame}
  \frametitle{Summary}
  We had a look at such topics as

  \begin{enumerate}
    \item General aspects of GHC and GHCi
    \item The Haskell Platform installation
    \item Dependency management using Stack and Cabal
    \item In other words, the Haskell ecosystem in a nutshell
  \end{enumerate}

  \vspace{\baselineskip}

\onslide<2->{
  On the next seminar, we will discuss:

  \begin{enumerate}
    \item The basic Haskell syntax
    \item The underlying aspects of the Haskell type system
    \item Functions and lambdas
    \item Immutability and Laziness
  \end{enumerate}
  }
\end{frame}

\end{document}
